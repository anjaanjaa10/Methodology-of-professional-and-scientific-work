\documentclass{article}
\usepackage[utf8]{inputenc}
\usepackage[serbian]{babel}
\usepackage{hyperref}
\usepackage{xcolor}
\usepackage{amsmath}
\usepackage{float}
\usepackage{pgf-pie}

\renewcommand{\contentsname}{Sadržaj}

\hypersetup{
    colorlinks=true, 
    linkcolor=blue,  
    urlcolor=blue 
}

\title{Istraživanje o korisnosti predavanja na Matematičkom fakultetu \\ \large Seminarski rad u okviru kursa \\ Metodologija stručnog i naučnog rada \\ Matematički fakultet}
\author{Anja Čolić, Ivana Nestorović, Jelena Lazović, Iva Đorđević}
\date{Novembar 2024}

\begin{document}

\maketitle

\begin{abstract}
Ovaj seminarski rad bavi se istraživanjem faktora koji čine predavanje korisnim, kroz anketu među studentima Matematičkog fakulteta. Cilj istraživanja bio je da se identifikuju ključni faktori koji doprinose kvalitetu nastave, kao i da se utvrdi koji oblici nastave olakšavaju razumevanje gradiva. Pored sprovođenja ankete, urađena je analiza dobijenih rezultata i njihovo poređenje sa relevantnom literaturom, kako bi se pružio dublji uvid u ovu temu.
\end{abstract}

\tableofcontents

\section{Uvod}

U okviru istraživanja o korisnosti predavanja na Matematičkom fakultetu, sprovedena je anketa sa ciljem da se sagledaju stavovi studenata o kvalitetu i obliku nastave i mogućnostima za unapređenje samih predavanja. Istraživanje je obuhvatilo \textbf{177 ispitanika}.

Ispitanici dolaze sa svih smerova i godina studija, što pruža širok spektar perspektiva i omogućava uvid u iskustva različitih grupa studenata. Istraživanje je obuhvatilo različite aspekte nastavnog procesa, uključujući klasična predavanja, vežbe, značaj vizuelnih materijala i praktičnih primera, kao i motivaciju studenata za dolazak na predavanja. Pored toga, studenti su izneli konkretne predloge za unapređenje nastave, koji će biti detaljno analizirani u daljem tekstu.

Cilj istraživanja bio je da se identifikuju ključni faktori koji utiču na korisnost predavanja i da se predlože mere za unapređenje kvaliteta nastave na Matematičkom fakultetu.
\section{Ciljevi istraživanja}
\begin{enumerate}
    \item \textbf{Poređenje različitih formata nastave} \\
Cilj je da se utvrdi koji format studenti smatraju najefikasnijim za usvajanje gradiva, kao i da se identifikuju prednosti i nedostaci svakog formata.
    \item \textbf{Faktori koji utiču na percepciju korisnosti predavanja medju studentima} \\
    Percepcija efikasnosti različitih nastavnih metodologija može se značajno razlikovati u zavisnosti od individualnih karakteristika studenata. Cilj je da se analizira kako faktori kao što su prisustvo predavanjima, godina studija, prosečna ocena, tip studija (osnovne ili master studije) i drugi lični faktori utiču na stavove studenata o različitim nastavnim metodama. 
    \item \textbf{Ispitivanje mišljenja studenata o ključnim karakteristikama korisnih predavanja} \\
    Korisna predavanja imaju odredjene karakteristike koje ih čine korisnim. Cilj je da se identifikuju osobine koje studenti smatraju najvažnijim za korisno predavanje, poput jasnog i zanimljivog prezentovanja gradiva, interaktivnosti, praktičnih primera i upotrebe tehnologije.
    \item \textbf{Procena uloge predavača u procesu učenja} \\
    Predavači igraju ključnu ulogu u obrazovnom procesu, jer način na koji prenose znanje i komuniciraju sa studentima direktno utiče na kvalitet nastave. Cilj je da se istraži kako studenti vide ulogu predavača, sa naglaskom na stručnost, komunikacijske veštine i sposobnost prilagođavanja potrebama studenata. Takođe, analizira se kako predavači mogu doprineti stvaranju atmosfere koja podstiče motivaciju, interakciju i aktivno učenje. \\
    \item \textbf{Motivacija za prisustvovanje predavanjima} \\
     Motivacija za prisustvovanje predavanjima igra ključnu ulogu u angažovanju studenata i njihovoj spremnosti da učestvuju u nastavi. Cilj ovog dela istraživanja je da se istraže faktori koji utiču na motivaciju studenata da prisustvuju predavanjima, kao što su relevantnost sadržaja, način izvođenja nastave, angažovanost predavača, kao i mogućnost interakcije tokom nastave, ali i socijalni aspekat. \\
     \item \textbf{Preporuke za unapređenje nastave} \\
     Na temelju dobijenih rezultata ovog istraživanja, predložiće se konkretne smernice za unapređenje nastave s ciljem poboljšanja kvaliteta obrazovnog procesa, povećanja zadovoljstva studenata i postizanja boljih akademskih performansi. 
\end{enumerate}

\section{Metodologija istraživanja}
Ovo istraživanje koristi kvantitativni pristup kako bi ispitalo stavove studenata Matematičkog fakulteta o različitim formatima nastave i faktorima koji utiču na njihovu percepciju korisnosti predavanja. Kao glavni metod za prikupljanje podataka korišćena je anketa.
\\
Podaci su prikupljeni putem online ankete koja je sadržala kombinaciju pitanja otvorenog i zatvorenog tipa. Pitanja su bila usmerena na ispitivanje stavova studenata prema različitim formatima nastave, poput klasičnih predavanja, vežbi, online predavanja u realnom vremenu i snimljenih predavanja. Takođe, anketa je uključivala pitanja o karakteristikama koje čine predavanja korisnim, kao što su uloga predavača, jasnoća objašnjavanja, praktična primenljivost gradiva, motivacija studenata i metode angažovanja tokom nastave.
\\
Anketa je distribuirana studentima putem studentskih grupa na društvenim mrežama i bila je dostupna za popunjavanje u periodu od tri dana. Anonimnost ispitanika je u potpunosti zagarantovana. Nakon zatvaranja ankete, prikupljeni podaci su obrađeni i analizirani, pri čemu su sprovedene statističke analize za kvantitativne odgovore, kao i tematska analiza odgovora otvorenog tipa. Rezultati su zatim uporedjeni sa relevantnom literaturom kako bi se pružio dublji uvid u rezultate i identifikovali ključni faktori korisnosti predavanja.
\section{Rezultati istraživanja}

U ovom delu rada analizirali smo odgovore ispitanika koji su učestvovali u anketi o korisnosti predavanja na Matematičkom fakultetu. Anketu je popunilo ukupno \textbf{177 ispitanika}, od kojih je \textbf{76.8\% na osnovnim studijama}, dok je preostalih \textbf{23.2\% na master studijama}. Od demografskih podataka smo takođe prikupili i informaciju o dužini i tipu studija, prosečnoj oceni, kao i zaposlenosti studenata. Na osnovu dobijenih podataka, \textbf{62.7\%} studenata je sa programskog studija \textbf{Informatika}, \textbf{36.2\%} sa studijskog programa \textbf{Matematika}, a \textbf{1.1\%} pohađa studijski program \textbf{Astronomija i astrofizika}. Kada je reč o godinama studija, najveći broj studenata je sa prve i četvrte godine, dok ostatak čine studenti ostalih godina, kao što je prikazano na slici~\ref{fig:duzinaStudija}.

Ovime smo omogućili dobijanje reprezentativnih rezultata koji odražavaju mišljenje studenata sa svih godina i svih smerova na fakultetu.

\begin{figure}[h!]
    \centering
    \includegraphics[width=1\linewidth]{Dužina studiranja_ (Koliko godina studirate na trenutnom programu_).png}
    \caption{Grafički prikaz dužine studija ispitanika}
    \label{fig:duzinaStudija}
\end{figure}
\subsection{Učestalost prisustvovanja predavanjima}
Jedno od ključnih pitanja u anketi odnosilo se na učestalost prisustvovanja predavanjima. Rezultati pokazuju da \textbf{45.2\% ispitanika redovno prisustvuje predavanjima}, \textbf{32.8\% povremeno}, dok su preostalih \textbf{22\% ispitanika} oni koji \textbf{retko ili uopšte ne prisustvuju nastavi}. Ovi podaci mogu ukazivati na različite faktore, poput opterećenja studentima ili organizacije nastave, koji mogu uticati na odluku o prisustvovanju predavanjima. Studentima koji redovno posećuju nastavu postavljeno je i pitanje o motivaciji dolaska na sama predavanja. Odgovori na ovo pitanje su uglavnom bili vezani za lakše savladavanje i razumevanje gradiva, pa samim tim i olakšano spremanje ispita kao i drženje sa kolegama.

\subsection{Zaposlenost ispitanika}
Anketirani studenti su takođe odgovarali na pitanje o njihovoj zaposlenosti. \textbf{80.2\%} ispitanika \textbf{nije zaposleno}, dok je preostalih \textbf{19.8\% zaposlenih}. Ovaj podatak može biti koristan u kontekstu analize uticaja zaposlenosti na angažman studenata u nastavi. Na primer, nezaposleni studenti mogu imati više vremena za redovno prisustvovanje predavanjima, što je u skladu sa prethodnim rezultatom o učestalosti prisustvovanja.

\subsection{Format predavanja}
Studenti generalno pokazuju veće zadovoljstvo vežbama i snimljenim predavanjima u odnosu na klasična i online predavanja uživo. Posebno je evidentno da snimljena predavanja pružaju najveći nivo fleksibilnosti i jasnoće, što ih čini omiljenim formatom. S druge strane, online predavanja uživo zahtevaju značajna unapređenja kako bi bolje zadovoljila potrebe studenata. Tabela\ref{tabela_format}.
\begin{table}[H]
\centering
\begin{tabular}{|l|c|c|c|}
\hline
\textbif{Format} & \textbf{Prosečna vrednost} & \textbf{Standardna devijacija} & \textbf{Medijana} \\
\hline
Predavanja & 3.27 & 1.05 & 3.0 \\
\hline
Vežbe & 3.89 & 0.97 & 4.0 \\
\hline
\parbox[t]{3cm}{Online predavanja} & 2.66 & 1.20 & 3.0 \\
\hline
\parbox[t]{3cm}{Online predavanja (snimljeno)} & 4.32 & 0.92 & 5.0 \\
\hline
\end{tabular}
\caption{Statistički podaci za različite formate nastave}
\label{tabela_format}
\end{table}

\subsection{Faktori koji doprinose korisnosti predavanja}
Istraživanje je identifikovalo ključne faktore koji utiču na korisnost predavanja prema mišljenju studenata. Faktori kao što su \textbf{način predavanja i posvećenost predavača, kvalitet materijala, kao i jasno objašnjenje složenih pojmova}, posebno se izdvajaju kao važni za većinu studenata. Detaljnije rezultate koji ilustruju ove faktore možete videti u tabeli \ref{tabela2_format} ispod.

\begin{table}[H] % Korišćenje [H] umesto [h!]
\centering
\begin{tabular}{|l|c|}
\hline
\textbf{Faktor} & \textbf{Broj studenata} \\
\hline
Način predavanja i posvećenost predavača & 154 \\
\hline
Podsticanje kritičkog razmišljanja i postavljanja pitanja & 84 \\
\hline
Način izvođenja predavanja (uživo u učionici ili online) & 57 \\
\hline
Interesovanje i motivacija koje predavač unosi u predavanje & 114 \\
\hline
Korišćenje zanimljivih primera i anegdota od strane predavača & 102 \\
\hline
Kvalitet materijala i prezentacija & 107 \\
\hline
Primeri iz prakse & 99 \\
\hline
Jasno objašnjenje složenih pojmova & 133 \\
\hline
\end{tabular}
\caption{Faktori koji najviše doprinose korisnosti predavanja}
\label{tabela2_format}
\end{table}
\subsection{Uloga predavača i način predavanja}
Podaci ukazuju na to da studenti generalno imaju pozitivno mišljenje o predavačima, naročito o njihovoj ulozi i organizaciji nastave. Međutim, postoje oblasti koje se mogu unaprediti, poput podsticanja aktivnog učešća studenata i češćeg korišćenja praktičnih primera u nastavi.

\begin{table}[H]
\centering
\begin{tabular}{|l|c|c|c|}
\hline
\textbf{Pitanje} &
  \textbf{Prosečna vrednost} &
  \textbf{Standardna devijacija} &
  \textbf{Medijana} \\ \hline
\parbox[t]{3cm}{Uloga predavača} &
  4.67 &
  0.63 &
  5.0 \\ \hline
\parbox[t]{3cm}{Predavanja su uglavnom organizovana tako da je gradivo lako pratiti} &
  3.03 &
  1.10 &
  3.0 \\ \hline
\parbox[t]{3cm}{Predavači podstiču aktivno učešće studenata tokom predavanja} &
  2.86 &
  1.00 &
  3.0 \\ \hline
\parbox[t]{3cm}{Vizuelni materijali (prezentacije, grafika) doprinose boljem razumevanju} &
  3.78 &
  1.11 &
  4.0 \\ \hline
\parbox[t]{3cm}{Predavači koriste primere iz prakse koji pomažu u razumevanju teorije} &
  3.20 &
  1.19 &
  3.0 \\ \hline
\end{tabular}
\caption{Rezultati ankete o ulozi predavača}
\end{table}
\subsection{Šta predavanje čini korisnim?}

Na osnovu analiziranih odgovora studenata, možemo identifikovati nekoliko ključnih tema koje se često pojavljuju. Ove teme nam pomažu da razumemo šta studenti smatraju važnim za korisnost predavanja:

\subsubsection{Kvalitet predavača}
Mnoge odgovore karakteriše fokus na sposobnost predavača da prenošenje znanja bude jasno i razumljivo. Studenti često spominju važnost interakcije sa studentima, postavljanja pitanja, davanja jednostavnih primera i objašnjavanja složenih pojmova na jednostavan način. Neki od ključnih odgovora su:
\begin{quote}
    \textit{"Predavanje je korisno ako predavač ima sposobnost da predaje"} \\
    \textit{"Predavanje je korisno ako se daju jednostavni primeri"} \\
    \textit{"Predavanje kroz koje profesor pokušava da objasni materiju na jednostavan način"}
\end{quote}

\subsubsection{Pristup i struktura predavanja}
Mnogi studenti navode da predavanje treba biti organizovano i da mora obuhvatiti osnovne pojmove, dok složeniji koncepti ne bi smeli biti previše zbunjujući. Takođe, postoji insistiranje na tome da predavanje ne bude samo čitanje sa slajdova, već treba uključivati dodatne informacije koje nisu prisutne u literaturi. Neki od ključnih odgovora su:
\begin{quote}
    \textit{"Korisno predavanje je ono na kom mogu da čujem stvari koje ne bih našla u literaturi"} \\
    \textit{"Predavanje koje ne podrazumeva samo čitanje sa prezentacije"} \\
    \textit{"Predavanje na kojem gradivo bude jasno i detaljno objašnjeno"}
\end{quote}

\subsubsection{Motivacija i interes}
Predavanje je korisno kada studenti osećaju da su motivisani, kada je sadržaj interesantan i kada predavač pokazuje strast za temom. Interesovanje studenata je ključ za održavanje njihove pažnje. Neki od ključnih odgovora su:
\begin{quote}
    \textit{"Predavanje je korisno kada uspeva da zainteresuje studenta za datu oblast"} \\
    \textit{"Korisno predavanje je interesantno i zanimljivo predavanje od strane predavača"}
\end{quote}

\subsubsection{Dostupnost i primenljivost}
Online predavanja i dostupnost materijala na internetu su često spominjani kao pozitivni aspekti koji omogućavaju studentima da prate gradivo čak i kada nisu fizički prisutni. Neki od ključnih odgovora su:
\begin{quote}
    \textit{"Online predavanja su veliki plus jer omogućavaju studentima da prate gradivo"} \\
    \textit{"Korisno predavanje omogućava lakše savladavanje gradiva"}
\end{quote}

\subsubsection{Kritičko razmišljanje i povezivanje sa stvarnim svetom}
Važno je da predavanje ne bude samo teorijsko, već da se povezuje sa realnim primerima i praktičnom primenom znanja. Neki od ključnih odgovora su:
\begin{quote}
    \textit{"Predavanje je korisno ako olakšava savladavanje gradiva"} \\
    \textit{"Predavanje kroz koje profesor skreće pažnju na bitnije stvari"}
\end{quote}

\subsubsection{Sentiment}

Generalni sentiment među studentima je uglavnom pozitivan prema korisnim predavanjima. Međutim, studenti su vrlo kritični prema neefikasnim metodama predavanja, poput monotonog čitanja sa prezentacija, što doprinosi negativnom stavu. Predavanja koja su dobro organizovana, interaktivna i motivirajuća izazivaju pozitivne reakcije.

Neki od najistaknutijih odgovora studenata uključuju:

\begin{quote}
    \textit{"Predavanje nije korisno ako se svodi na citanje pdf/prezentacije monotonim glasom bez dubljeg razumevanja."}
\end{quote}

\begin{quote}
    \textit{"Korisno predavanje je ono na kojem gradivo bude jasno i detaljno objašnjeno, posle kojeg ne moram da tumačim dokaze i pojmove sama kod kuće."}
\end{quote}

\begin{quote}
    \textit{"Predavanje je korisno ako predavač ume da održi pažnju, ukoliko jasno priča, daje primere koji su lako shvatljivi."}
\end{quote}

\begin{quote}
    \textit{"Predavanje je korisno kada uspeva da zainteresuje studenta za datu oblast."}
\end{quote}

\begin{quote}
    \textit{"Korisno predavanje je interesantno i zanimljivo, predavač pre svega voli to što radi i prenosi znanje."}
\end{quote}
\subsubsection{Najčešće pominjani kursevi}
Iz odgovora je moguće identifikovati nekoliko kurseva koji su među studentima najviše cenjeni zbog korisnosti predavanja. Studenti su kurs smatrali korisnim iz sledećih razloga:

\begin{enumerate}
    \item \textbf{Kvalitet predavača}: Većina odgovora ukazuje na značajan uticaj profesora na percepciju korisnosti kursa. 
    \textit{Primeri:} 
    \begin{itemize}
        \item \emph{„Algoritmi i strukture podataka kod Filipa Marića – predavač je učinio gradivo zanimljivim i jasno ga objasnio.“}
        \item \emph{„Diferencijalne jednačine kod Marije Mikić – predavanja su interaktivna i motivišu studente.“}
    \end{itemize}

    \item \textbf{Interaktivnost predavanja}: Mnogi studenti su istakli značaj interakcije tokom predavanja, bilo kroz postavljanje pitanja, diskusije, ili rešavanje problema na licu mesta.  
    \textit{Primer:} 
    \begin{itemize}
        \item \emph{„Konstrukcija i analiza algoritama – gradivo je obrađivano uz interakciju sa studentima, što je značajno olakšalo razumevanje.“}
    \end{itemize}

    \item \textbf{Praktični primeri}: Kursevi koji povezuju teoriju sa praktičnim primerima često se ocenjuju kao korisni.  
    \textit{Primer:} 
    \begin{itemize}
        \item \emph{„Veštačka inteligencija kod Predraga Janičića – predavanja su obuhvatala primere iz svakodnevnog života koji olakšavaju razumevanje.“}
    \end{itemize}

    \item \textbf{Podrška i resursi za učenje}: Kursevi koji nude jasno definisane resurse, kao što su materijali za učenje, video-snimci predavanja, i primeri, takođe su veoma cenjeni.  
    \textit{Primer:} 
    \begin{itemize}
        \item \emph{„Programiranje (1, 2, i drugi kursevi) – dostupnost materijala značajno olakšava učenje.“}
    \end{itemize}

    \item \textbf{Dugotrajni utisak gradiva}: Kursevi koji su ostavili snažan utisak na studente i posle nekoliko godina se izdvajaju kao korisni.  
    \textit{Primer:} 
    \begin{itemize}
        \item \emph{„Veštačka inteligencija kod prof. Janičića – gradivo se pamti godinama zahvaljujući zanimljivom načinu predavanja.“}
    \end{itemize}
\end{enumerate}

\section{Zanimljive činjenice}

U okviru istraživanja analizirane su korelacije između prosečnih ocena studenata i učestalosti njihovog prisustva na predavanjima, kao i korelacija između godine studija i učestalosti prisustva. Rezultati ove analize pomažu u razumevanju kako prisustvo na predavanjima može uticati na akademski uspeh i kako se obrazovni proces menja tokom studija.

\subsection{Korelacija prosečne ocene i prisustva na predavanjima}

Tabela \ref{tab:korelacija_ocena_prisustvo} prikazuje korelaciju između prosečnih ocena i učestalosti prisustva na predavanjima. Rezultati ukazuju da studenti koji redovno dolaze na predavanja najčešće ostvaruju bolje prosečne ocene. Posebno se vidi da redovno prisustvo pozitivno utiče na postizanje prosečnih ocena između 8.00 i 8.99, dok povremeno prisustvo dominira kod studenata koji imaju prosečne ocene između 7.00 i 7.99. Ovi podaci ukazuju na važnost redovnog prisustva u postizanju visokih akademskih rezultata.

\begin{table}[h!]
\centering
\begin{tabular}{|c|c|c|c|c|}
\hline
\textbf{Prosečna ocena} & \textbf{Nikad} & \textbf{Povremeno} & \textbf{Redovno} & \textbf{Retko} \\ \hline
6.00--6.99 & 1 & 1 & 2 & 3 \\ \hline
7.00--7.99 & 5 & 22 & 18 & 14 \\ \hline
8.00--8.99 & 0 & 12 & 18 & 3 \\ \hline
9.00--10.00 & 4 & 4 & 7 & 2 \\ \hline
\end{tabular}
\caption{Korelacija prosečne ocene i prisustva na predavanjima}
\label{tab:korelacija_ocena_prisustvo}
\end{table}

\subsection{Korelacija dužine studija i prisustva na predavanjima}

Tabela \ref{tab:korelacija_studiji_prisustvo} prikazuje korelaciju između godine studija i učestalosti prisustva na predavanjima. Sa napredovanjem kroz studije, studenti postaju sve manje redovni u prisustvovanju predavanjima. U prvoj i drugoj godini, studenti su najposvećeniji prisustvovanju, dok se tokom vremena beleži pad u redovnom prisustvu. Na kraju studija, većina studenata izbegava dolazak na predavanja, što ukazuje na smanjenje angažmana kako studije odmiču.

\begin{table}[h!]
\centering
\begin{tabular}{|c|c|c|c|c|}
\hline
\textbf{Godina studija} & \textbf{Nikad} & \textbf{Povremeno} & \textbf{Redovno} & \textbf{Retko} \\ \hline
I godina & 0 & 6 & 26 & 2 \\ \hline
II godina & 0 & 3 & 11 & 2 \\ \hline
III godina & 2 & 6 & 3 & 2 \\ \hline
IV godina & 1 & 12 & 17 & 3 \\ \hline
V godina & 2 & 8 & 5 & 9 \\ \hline
VI godina & 0 & 5 & 5 & 1 \\ \hline
VII godina & 1 & 1 & 2 & 0 \\ \hline
VIII godina & 0 & 1 & 0 & 0 \\ \hline
\end{tabular}
\caption{Korelacija dužine studija i prisustva na predavanjima}
\label{tab:korelacija_studiji_prisustvo}
\end{table}
Ovi podaci mogu poslužiti kao smernice za buduće obrazovne strategije i poboljšanje angažovanja studenata.

\subsection{Sugestije za poboljšanje nastave}
Kao deo ankete, studenti su imali priliku da iznesu svoje sugestije za poboljšanje nastave. Najčešće spomenute teme bile su:
\begin{itemize}
    \item \textbf{Povećanje interaktivnosti}: Mnogi studenti su naglasili da bi želeli više praktičnih primera i interaktivnih elemenata tokom predavanja.
    \item \textbf{Bolja organizacija nastave}: Studenti su sugerisali bolje usklađivanje termina predavanja i veći broj konsultacija.
    \item \textbf{Korišćenje vizuelnih materijala}: Bilo je nekoliko sugestija za učestaliju upotrebu vizuelnih materijala, kao što su prezentacije i video materijali, što bi moglo poboljšati razumljivost sadržaja.
\end{itemize}
Ove sugestije mogu biti korisne za buduća unapređenja u organizaciji nastave i metodologiji.



\section{Poređenje rezultata istraživanja i teorijskih okvira}

Rezultati ankete pokazuju da studenti Matematičkog fakulteta ističu dobro objašnjene primere i praktičnu primenljivost gradiva kao ključne faktore korisnog predavanja. Prema Mayeru (2009), kombinovanje verbalnih i vizuelnih elemenata doprinosi dubljoj obradi informacija, što podržava ovakve stavove studenata. Aktivne metode, poput diskusija i rešavanja problema, koje Bonwell \& Eison (1991) opisuju kao transformativne za proces učenja, dodatno osnažuju motivaciju studenata.

Studenti su istakli da interaktivne aktivnosti, kao što su postavljanje pitanja, predviđanje rezultata, i primena anketa ili kvizova, značajno olakšavaju razumevanje gradiva. Na primer, korišćenje alata poput Kahoota ili Mentimetera omogućava trenutnu povratnu informaciju i podstiče angažovanost. Nicol \& Macfarlane-Dick (2006) naglašavaju da ovakve aktivnosti pomažu studentima da identifikuju slabosti u razumevanju i da unaprede proces učenja.

Kvalitet predavanja često zavisi i od entuzijazma i prezentacijskih veština predavača. Studenti su istakli da energija i jasnoća predavača inspirišu njihovo interesovanje i motivaciju, što je u skladu sa tvrdnjama Brookfielda (2015) u delu *The Skillful Teacher*. Pored toga, socijalna dimenzija predavanja, uključujući interakciju sa kolegama, povećava interes za prisustvovanje i doprinosi boljem razumevanju gradiva (Kuh et al., 2005).

Motivacija studenata da prisustvuju predavanjima proizilazi iz vrednosti koje im ona pružaju – priprema za ispite, dublje razumevanje gradiva i dodatne informacije koje nisu dostupne u udžbenicima. Ambrose i saradnici (2010) ističu da upravo praktična primenljivost i relevantnost sadržaja povećavaju vrednost predavanja u očima studenata.

Poređenjem rezultata istraživanja sa teorijskim okvirima može se zaključiti da su stavovi studenata i preporuke iz literature u velikoj meri saglasni. Naglasak na kvalitetnom objašnjavanju, aktivnim metodama učenja i povezivanju teorije i prakse potvrđuje značaj predavanja kao ključnog elementa savremenog obrazovanja.


\section{Zaključak}

Istraživanje među studentima Matematičkog fakulteta pokazalo je da kvalitet predavanja zavisi od nekoliko ključnih faktora: jasna i praktična objašnjenja, dobra prezentacija gradiva i primena aktivnih metoda učenja. Studenti su istakli da im je lakše razumeti gradivo kada su predavanja dobro osmišljena i kada uključuju konkretne primere i praktičnu primenljivost. Kao dodatni važan element, većina studenata izrazila je preferenciju za dostupnost snimljenih materijala, koji im omogućavaju da gradivo savladaju sopstvenim tempom i po potrebi ponove predavanja.
\\ \\
Motivacija za dolazak na predavanja najčešće je povezana sa lakšim razumevanjem gradiva, socijalnim aspektima nastave i pripremom za ispite. Poređenje rezultata ankete sa stručnom literaturom potvrdilo je značaj kombinovanja vizuelnih i verbalnih elemenata, upotrebe interaktivnih alata kao što su ankete i kvizovi, i podsticanja kritičkog razmišljanja kroz aktivno učešće. Dobijeni rezultati su u skladu sa savremenim pedagoškim preporukama, koje ukazuju na važnost prilagodjavanja nastave potrebama studenata.
\\ \\
Na osnovu rezultata, preporučuje se unapredjenje nastavnih strategija koje uključuju jasniju prezentaciju gradiva, veći stepen interakcije tokom nastave i uvodjenje snimljenih materijala kao standardne prakse. Ovi koraci mogu značajno doprineti stvaranju fleksibilnijeg i dinamičnijeg obrazovnog okruženja koje motiviše studente i pomaže im u efikasnijem učenju.
\section{Literatura}
\begin{enumerate}
    \item Mayer, R. E. (2009). \textit{Multimedia Learning.}.
    \item Bonwell, C. \& Eison, J. (1991). \textit{Active Learning: Creating Excitement in the Classroom.}
    \item Ambrose, S. A., et al. (2010). \textit{British Journal of Educational Psychology.}
    \item Brookfield, S. D. (2015). \textit{The Skillful Teacher: On Technique, Trust, and Responsiveness in the Classroom.} 
    \item Kuh, G. D., et al. (2005). \textit{Student Success in College: Creating Conditions That Matter.}
    \item Kolb, D. A. (1984). \textit{Experiential Learning: Experience as the Source of Learning and Development.}
        \item Nicol, D. J., \& Macfarlane-Dick, D. (2006).     \textit{Formative Assessment and Self-Regulated Learning: A Model and Seven Principles of Good Feedback Practice.}\href{https://www.researchgate.net/publication/228621906_Formative_Assessment_and_Self-Regulated_Learning_A_Model_and_Seven_Principles_of_Good_Feedback_Practice}{www.researchgate.net}
\end{enumerate}

\end{document} 